\subsection{Hypotheses}\label{hypotheses}

In addition to the theoretical model presented above, we do some hypotheses due to the fact that we place ourselves in a particular frame of study, namely the study of M/EEG data.
This therefore involves making some specific assumptions in order to take into account the physical reality that applies.
In addition, certain assumptions facilitate the calculations, without prejudice to the veracity of the model.
Those hypotheses are as follow.


\begin{enumerate}
    \item In Equation~\eqref{eq:model_intensity}, the kernel's purpose is to capture the delay of the neurological response following an external stimulation.
    This kernel is chosen to be a truncated normal of shape parameter $m$ and $\sigma$, respectively representing the mean and the standard deviation of such a delay.
    Precisely, we suppose that we have:
    \begin{equation}\label{eq:m_in_between}
        a \leq m \leq b
    \end{equation}
    where $a$ and $b$ are respectively the lower and upper truncation values.
    
    \item The shape parameters of the kernel depend on the couple atom/task, i.e. $m\equiv m_{k,p}$ and $\sigma \equiv \sigma_{k,p}$, in order to better capture the specificity of a task and its neurological response (in time and in space).
    
    \item The truncation values $a$ and $b$ are the same for every atom $k$ and every task $p$.
    The idea behind this hypothesis is that the neurological response cannot, in any case, be shorter than $a$ (e.g., $a = \SI{30e-3}{\second}$) nor be longer than $b$ (e.g., $b = \SI{800e-3}{\second}$)
    
    \item The tasks are far enough apart from each other such that the kernel is null right before every tasks, i.e.:
    \begin{equation}\label{eq:separated_timestamps}
        t_{i+1}^{(p)} - t_i^{(p)} > b, \quad \forall p=1,\dots,P, \quad \forall i=1,\dots, (n_p-1)
    \end{equation}
    In the M/EEG context, we say that the \textit{inter stimulus interval} (ISI) is larger than $b$.
    In consequence, unlike the Hawkes processes model with exponential kernels where all past events influence the intensity, in our model only the last stimulus at time $t$ has a possible effect on the intensity at that time.
    
    Note that with this hypothesis, the intensity can be rewritten similar as a auto-regressive point process:
    \begin{equation}
        \lambda_{k,p}(t)  = \mu_k + \alpha_{k,p}\sum_{t_i^{(p)} \leq t} \kappa_{k,p}(t - t_i^{(p)})
    \end{equation}
    indeed, as $t \geq t_*^{(p)}(t) > \dots > t_2^{(p)} > t_1^{(p)}$, then $\forall i, t_i^{(p)} < t_*^{(p)}(t)$
    \begin{align*}
        t - t_i^{(p)} &= t - t_{i+1}^{(p)} + t_{i+1}^{(p)} - t_i^{(p)} \\
        &> t - t_{i+1}^{(p)} + b \\
        &\geq b, \quad \text{as } t - t_{i+1}^{(p)}\geq 0
    \end{align*}
    and as by definition, $\forall x \geq b, \kappa_{k,p}(x)=0$, thus $\kappa_{k,p}(t - t_1^{(p)}) = 0$, and finally,
    \begin{align*}
        \sum_{t_i^{(p)} \leq t} \kappa_{k,p}(t - t_i^{(p)}) &= \kappa_{k,p}(t - t_1^{(p)}) + \kappa_{k,p}(t - t_2^{(p)}) + \dots + \kappa_{k,p}(t - t_*^{(p)}(t)) \\
        &= 0 + 0 + \dots + \kappa_{k,p}(t - t_*^{(p)}(t))
    \end{align*}
    
    \item The experimentation ends after every possible neurological response driven by a task, i.e.,
    \begin{equation}\label{eq:long_duration}
        \forall p=1,\dots,P, \quad T > t_{n_p}^{(p)} + b
    \end{equation}
    $$
    \Leftrightarrow T > \maxx{p=1,\dots,P} t_{n_p}^{(p)} + b
    $$
    \item An atom is driven by only one task, i.e., we only consider the effect of one task for an atom in the model, as opposed to including several tasks.
    
    \item The baseline intensity $\mu_k$ is fixed in time.
    The baseline coefficient aims at capturing every exogenous sources of activation that are not controlled by the model, i.e., all other sources of activation apart from the considered task.
    Other sources can be other tasks, other atoms, spontaneous activations or even sources we are not aware of.
    Finally, it should be pointed out that, unlike Hawkes processes, we do not model any self-excitatory behaviour, i.e., an activation of the considered atom does not increase by itself the probability to have another activation.
\end{enumerate}

These assumptions can easily be called into question, in particular their restrictive character on the control of endogenous sources of activation, as we only consider an atom along a unique task.
That is why they we be discussed later on, and a brief overview of an extension of our model will be presented in Section~\ref{model_extension_multiple_pp}.
