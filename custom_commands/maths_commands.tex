%%%%%%%%%%%%%%%%%%%%%%%%%%%%%%%%%%%%%%%%%%%%%%%%
% Ce fichier contient des nouvelles commandes 
% pour faciliter l'écriture des mathématiques
%%%%%%%%%%%%%%%%%%%%%%%%%%%%%%%%%%%%%%%%%%%%%%%%

% Ensembles
\let\ensembleNombre\mathbb
\newcommand*\N{\ensuremath{\ensembleNombre{N}}}       % Ensemble des entiers naturels N
\newcommand*\Z{\ensuremath{\ensembleNombre{Z}}}       % Ensemble des entiers relatifs Z
\newcommand*\Q{\ensuremath{\ensembleNombre{Q}}}       % Ensemble des rationnels Q
\newcommand*\R{\ensuremath{\ensembleNombre{R}}}       % Ensemble des réels R
\newcommand*\Complex{\ensuremath{\ensembleNombre{C}}} % Ensemble des complexes C

% Encadrement
\newcommand{\angles}[1]{\left\langle #1 \right\rangle} % ⟨⟩
\newcommand{\braces}[1]{\left\lbrace #1 \right\rbrace} % {}
\newcommand{\bracks}[1]{\left\lbrack #1 \right\rbrack} % []
\newcommand{\pars}[1]{\left( #1 \right)}               % ()
\newcommand{\norme}[1]{\left\Vert #1\right\Vert}       % Norme
\newcommand{\ent}[1]{\left\lfloor #1\right\rfloor}     % Partie entière inférieure
\newcommand{\entsup}[1]{\left\lceil #1\right\rceil}    % Partie entière supérieure
\newcommand{\abs}[1]{\left\vert #1\right\vert}         % Valeur absolue

% Dérivées
\newcommand{\deriv}[3][]{\frac{\mathrm{d}^{#1}#2}{\mathrm{d} {#3}^{#1}}}    % Dérivée non partielle  
\newcommand{\partiald}[3][]{\frac{\partial^{#1}#2}{\partial {#3}^{#1}}} % Dérivée partielle   

% Intervalles
\newcommand*\intervalle[4]{\left#1 #2 \, ; #3 \right#4}                       % Définit la notion d'intervalle
\newcommand*\intervalleOO[2]{\intervalle{(}{#1}{#2}{)}}                       % ]a ; b[
\newcommand*\intervalleFF[2]{\intervalle{[}{#1}{#2}{]}}                       % [a ; b]
\newcommand*\intervalleOF[2]{\intervalle{(}{#1}{#2}{]}}                       % ]a ; b]
\newcommand*\intervalleFO[2]{\intervalle{[}{#1}{#2}{)}}                       % [a ; b[
\newcommand*\intervalleEntier[2]{\intervalle{\llbracket}{#1}{#2}{\rrbracket}} % Ensemble d'entiers

% Ensemble "tel que"
\newcommand*\enstq[2]{\braces{#1,\; #2}}                                      

% Probabilité, espérance, variance, covariance et corrélation
\newcommandx{\proba}[3][1, 3=0]{
\ifthenelse{\isempty{#2}}{\mathbb{P}_{#1}}{%
\ifthenelse{\equal{#3}{0}}{\mathbb{P}_{#1}\pars{#2}}{\mathbb{P}_{#1}\pars{#2 \middle| #3}}}%
}
\newcommandx{\esp}[3][1, 3=0]{
\ifthenelse{\isempty{#2}}{\mathbb{E}_{#1}}{%
\ifthenelse{\equal{#3}{0}}{\mathbb{E}_{#1}\bracks{#2}}{\mathbb{E}_{#1}\bracks{#2 \middle| #3}}}%
}
\newcommandx{\var}[3][1, 3=0]{
\ifthenelse{\isempty{#2}}{\operatorname{Var}_{#1}}{%
\ifthenelse{\equal{#3}{0}}{\operatorname{Var}_{#1}\pars{#2}}{\operatorname{Var}_{#1}\pars{#2 \middle| #3}}}%
}
\newcommandx{\cov}[4][1, 4=0]{
\ifthenelse{\isempty{#2} \AND \isempty{#3}}{\operatorname{Cov}_{#1}}{%
\ifthenelse{\equal{#4}{0}}{\operatorname{Cov}_{#1}\pars{#2, #3}}{\operatorname{Cov}_{#1}\pars{#2, #3 \middle| #4}}}%
}
\newcommandx{\corr}[3][1]{\ifthenelse{\isempty{#2} \AND \isempty{#3}}{\operatorname{Corr}_{#1}}{\operatorname{Corr}_{#1}\pars{#2, #3}}}

% Lois usuelles de probabilités
\newcommand{\Ber}[1]{\mathcal{B}ernoulli\pars{#1}}
\newcommand{\Binom}[2]{\mathcal{B}\pars{#1, #2}}
\newcommand{\Poisson}[1]{\mathcal{P}\pars{#1}}
\newcommand{\Geo}[1]{\mathcal{G}eo\pars{#1}}
\newcommand{\Unif}[1]{\mathcal{U}nif_{#1}}
\newcommand{\Exp}[1]{\mathcal{E}xp\pars{#1}}
\newcommandx{\Norm}[3][1]{\mathcal{N}_{#1}\pars{#2, #3}}
\newcommand{\Gam}[2]{\Gamma\pars{#1, #2}}
\newcommand{\InvGam}[2]{\text{Inv-}\Gamma\pars{#1, #2}}
\newcommand{\Betaloi}[2]{\beta\pars{#1, #2}}
\newcommand{\Pareto}[2]{\mathcal{P}areto\pars{#1, #2}}
\newcommand{\Laplace}[1]{\mathcal{L}\pars{#1}}
\newcommand{\Cauchy}[1]{\mathcal{C}\pars{#1}}

% Égalités
\newcommandx{\egalloi}{\overset{\mathrm{\mathcal{L}oi}}{=}}
\newcommandx{\egalprob}{\overset{\mathbb{P}}{=}}
\newcommandx{\egaltxt}[1]{\overset{\mathrm{#1}}{=}}
\newcommandx{\simiid}{\overset{\mathrm{iid}}{\sim}}

% Indicatrice
\newcommand{\1}[1][]{\mathds{1}_{\braces{#1}}}

% max, min, argmax et argmin avec subscript
\newcommandx{\maxx}[1]{\underset{#1}{\max}}
\newcommandx{\minn}[1]{\underset{#1}{\min}}
\newcommandx{\argmax}[1][1]{\underset{#1}{\arg\max}}
\newcommandx{\argmin}[1][1]{\underset{#1}{\arg\min}}

% matrice 3*3 diagonale avec la diagonale donnée sur les 3 arguments
\newcommand{\mattroisdiag}[3]{\left( \begin{array}{ccc} #1 &  0 &  0 \\ 0  & #2 &  0 \\ 0  &  0 & #3\end{array} \right)}

% Intégration
\newcommand{\dint}{\mathrm{d}}
\newcommandx{\integ}[4][2]{\int_{#1}^{#2} #3 \, \mathrm{d} #4} 

% moyenne simple et quadratique d'une grandeur scalaire
\newcommand{\moy}[1]{\overline{#1}}
\newcommand{\moyy}[1]{\overline{#1^2}}

% Vecteurs
\let\vecteur\overrightarrow % Permet de modifier la taille de la flèche d'un vecteur quand ce dernier est composé de plusieurs lettres par exemple.
\newcommand*\coord[2]
{
   \vecteur{#1} \,
   \begin{pmatrix}
      #2
   \end{pmatrix}
} % Permet de déclarer les coordonnées d'un vecteur, exemple : \[\coord{AB}{12\\21} \coord{BC}{21\\12\\32}\]

% Surface et volume
\newcommandx{\surf}[2][2=m]{\numprint{#1}\,#2\textsuperscript{2}}
\newcommandx{\vol}[2][2=m]{\numprint{#1}\,#2\textsuperscript{3}}

% Fonction
%\newcommand*\fonction[5]{
%#1 \colon \left\{\begin{alignedat}{2}  &#2 &\: &\to      #3\\
%                                &#4 &   &\mapsto  #5
%\end{alignedat} \right. \kern-\nulldelimiterspace} % Permet de déclarer facilement une fonction, exemple : $\fonction{f}{E}{F}{x}{f(x)}$

\newcommand*\fonction[5]{\begin{align*}#1 \colon #2 & \to #3\\#4 & \mapsto #5\end{align*}} 
% Permet de déclarer facilement une fonction, exemple : $\fonction{f}{E}{F}{x}{f(x)}$

\newcommand{\twopartdef}[4]
{
	\left\{
		\begin{array}{ll}
			#1 & \mbox{si } #2 \\
			#3 & \mbox{si } #4
		\end{array}
	\right.
}

\newcommand{\threepartdef}[6]
{
	\left\{
		\begin{array}{lll}
			#1 & \mbox{si } #2 \\
			#3 & \mbox{si } #4 \\
			#5 & \mbox{si } #6
		\end{array}
	\right.
}

% Autres commandes utiles
\newcommand{\compos}{\circ}                 % operateur de composition de fonctions
\DeclareMathOperator{\card}{card}           % On déclare l’opérateur \card qui correspond à « card ».
\newcommand*\PI{\ensuremath{\pi}}           % Permet d'afficher pi que l'on soit en mode math ou non.
\newcommand{\ds}[1]{\displaystyle{#1}}      % Force LaTeX à gérer les indices et les exposants comme si il était en mode mathématique isolé
% fonction exponentielle
\newcommand{\e}[1]{\exp{\pars{#1}}}

\newcommand{\kronecker}[0]{\mathrm{\delta}} % symbole de Kronecker
\newcommand{\heaviside}[0]{\mathrm{H}}      % fonctions d'Heaviside
\newcommand{\dirac}[0]{\mathrm{\delta}}     % Distribution de Dirac

\newcommand*{\TakeFourierOrnament}[1]{{%
\fontencoding{U}\fontfamily{futs}\selectfont\char#1}}
\newcommand*{\danger}{\TakeFourierOrnament{66}}