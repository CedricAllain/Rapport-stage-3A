\section{Developed method: driven temporal point processes}\label{developed_method}

In the Hawkes processes model previously presented in Section~\ref{hawkes_processes}, the assumption is made that all the nodes might have an influence on all the other nodes, in addition of themselves.
Recall that in our application to electrophisiology, and in particular to M/EEG data, we would like to capture the potential influence of an external stimulus on the activation of a specific recurring pattern in the cortex, called \textit{atom}.
It is easy to understand in that case that the influence can only goes one way, from the stimulus process to the atom process.
Furthermore, the stimulus process is determinist as it is controlled by the experimentator and thus cannot be modeled as a classical temporal point process.
If we would like to implement such a model using \texttt{Tick}, similarly as the Figure~\ref{fig:tick_hawkes_model}, three of the four kernels would be set to the null function, as the assumption is made that the atom do not have a self-excitatory behaviour\footnote{More details on the assumptions made later in Section~\ref{hypotheses}.}, thus only leaving the kernel connecting the stimulus to the atom as a non-null function.

For those reasons, we derived a model from the Hawkes processes model, in which a determinist point process has a potential influence on a temporal point process.
By \say{having an influence on}, we mean that an event on the determinist point process increases the probability to have an event on the temporal point process.
We call this approach \textit{driven temporal point processes}, as the behaviour of a temporal point process might be driven, influenced, by a determinist point process.

\subsection{Data and notations}

TODO : Faire la distinction entre le modèle théorique (processus guidé et processus "guideur"), et l'application : les z\_k sont les processus guidés, et dans l'application on les obtient de telle manière [...] 

\begin{itemize}
    \item $\braces{z_k}_{k=1,\dots,K}$: the family of the $K$ atoms' activation vector.
    $\forall k=1,\dots,K, z_k\in\R_+^\mathbf{T}$, where $\mathbf{T}$ is the total number of possible timestamps.
    $\mathbf{T} \coloneqq T \times s_{freq}$, where $T$ is the total duration of the experimentation, and $s_{freq}$ is the sampling frequency for the collect of MEG data.
    We define $\mathcal{A}_k \coloneqq \enstq{t\times s_{freq}^{-1}}{z_k(t)\geq\tau, t=1,\dots, \mathbf{T}}$ the set of all the timestamps of atom $k$'s activations, where $\tau$ is a pre-determined threshold used to filter out insignificant activations (e.g. $\tau = \num{0.7e-10}$).
    
    \item $\braces{e_p}_{p=1,\dots,P}$: the family of the $P$ types of tasks performed during the experimentation
    \begin{equation*}
        \forall t=1,\dots,\mathbf{T}, \quad e_p(t)=
        \left\{
		\begin{array}{ll}
			1 & \mbox{if a task of type $p$ has occurred at the $t$-th timestamp}\\
			0 & \mbox{otherwise}
		\end{array}
	\right.
    \end{equation*}
    
    \item $t^{(p)} \coloneqq \enstq{t\times s_{freq}^{-1}}{e_p(t)=1, t=1,\dots, \mathbf{T}}$ denotes all the timestamps when a task of type $p$ occurred.
    We thus have $t^{(p)} = \braces{t^{(p)}_1, \dots, t^{(p)}_{n_p}}$ sorted, i.e., $t^{(p)}_1 < t^{(p)}_2 < \dots < t^{(p)}_{n_p}$, where $n_p=\# t^{(p)}$ is the total number of task $p$ that has occurred during the experimentation, as $\# A$ denotes the cardinality of the set $A$.
\end{itemize} 
\subsection{Model with truncated gaussian kernel}

In this section is now presented the choice of the model made to study the possible influence of a determinist point process $e_p$ on a temporal point process $z_k$, where $e_p$ and $z_k$ correspond to specific data in the context of electrophysiology, as mentioned above.
Several choices and assumptions are made in order to obtain the model presented below, those are detailed and explained later in Section~\ref{hypotheses}, in particular concerning the choice of the kernel shape.

When driven by $e_p$, the dynamic of $z_k$'s counting process $N_t^{(k)} \coloneqq \sum_{t_k \in \mathcal{A}_k} \1[t_k \leq t]$ can be describe by its intensity: $\forall t\in\intervalleFF{0}{T}$,
\begin{equation}\label{eq:intensity_definition}
    \lambda_{k,p}(t) \equiv \lambda_{k}\pars{t \middle| \mathscr{F}_t^{(p)}} \coloneqq \lim_{\dint t \xrightarrow{} 0 }\frac{\proba{N_{t+\dint t}^{(k)} - N_t^{(k)} = 1}[\mathscr{F}_t^{(p)}]}{\dint t}
\end{equation}
where $\mathscr{F}_t^{(p)} = \enstq{t_i \in t^{(p)}}{t_i < t}$ is the information available on $e_p$ up to, but not including, $t$.

\paragraph{Our model} As previously mentioned, the shape of the intensity is derived from the Hawkes process model, with a baseline intensity and a kernel function.
Note that here again, we consider the link function to be the identity function.
Namely,

\begin{equation}\label{eq:model_intensity}
    \lambda_{k,p}(t)  = \mu_k + \alpha_{k,p}\kappa_{k,p}(t - t_*^{(p)}(t))\1[t \geq t_1^{(p)}]
\end{equation}
where $t_*^{(p)}(t) \coloneqq \max\enstq{t'}{t' \in t^{(p)}, t' \leq t}$ denotes the timestamp of the last task of type $p$ occurred before ($\leq$) time $t$, $\alpha_{k,p}\in\R$ is a coefficient determining the relative importance of the stimulus $p$ in the activation of the atom $k$, and where $\kappa_{k,p}$ is the probability density function of a truncated gaussian distribution of mean $m_{k,p}\in\R$ and standard deviation $\sigma_{k,p}>0$, with truncation values $\intervalleFF{a}{b}\in\R$, noted $\Norm[\bracks{a,b}]{m_{k,p}}{\sigma_{k,p}^2}$, namely,
\begin{equation}\label{eq:kernel_trunc_norm}
    \kappa_{k,p}(x) \equiv \kappa(x ; m_{k,p}, \sigma_{k,p}, a, b)=\frac{1}{\sigma_{k,p}} \frac{\phi\left(\frac{x-m_{k,p}}{\sigma_{k,p}}\right)}{\Phi\left(\frac{b-m_{k,p}}{\sigma_{k,p}}\right)-\Phi\left(\frac{a-m_{k,p}}{\sigma_{k,p}}\right)} \1[a\leq x\leq b]
\end{equation}
where
$$
\phi(x) = \frac{1}{\sqrt{2\pi}}\exp\left(-\frac{1}{2}x^2\right)
$$ 
is the probability density function of the standard normal distribution, and
$$
\Phi(x) = \frac{1}{\sqrt{2\pi}} \int_{-\infty}^x \exp\left(-\frac{1}{2}t^2\right) \mathrm{d}t
$$
is its cumulative distribution function.

Note that by definition, if $b=\infty$, then $\Phi\left(\frac{b-m_{k,p}}{\sigma_{k,p}}\right) = 1$, and similarly, if $a=-\infty$, then $\Phi\left(\frac{a-m_{k,p}}{\sigma_{k,p}}\right) = 0$.
An example of a truncated normal distribution is presented in Fig.~\ref{fig:trunc_norm_distribution}.

\begin{figure}[h!]
    \centering
    \includegraphics[width=0.9\textwidth]{pics/trunc_norm_distribution.pdf}
    \caption{Truncated Normal distribution with $m=\num{0.2}$, $\sigma=0.1$, $a=0.03$ and $b=0.8$.}
    \label{fig:trunc_norm_distribution}
\end{figure}
\subsection{Hypotheses}\label{hypotheses}

In addition to the theoretical model presented above, we do some hypotheses due to the fact that we place ourselves in a particular frame of study, namely the study of M/EEG data.
This therefore involves making some specific assumptions in order to take into account the physical reality that applies.
In addition, certain assumptions facilitate the calculations, without prejudice to the veracity of the model.
Those hypotheses are as follow.


\begin{enumerate}
    \item In Equation~\eqref{eq:model_intensity}, the kernel's purpose is to capture the delay of the neurological response following an external stimulation.
    This kernel is chosen to be a truncated normal of shape parameter $m$ and $\sigma$, respectively representing the mean and the standard deviation of such a delay.
    Precisely, we suppose that we have:
    \begin{equation}\label{eq:m_in_between}
        a \leq m \leq b
    \end{equation}
    where $a$ and $b$ are respectively the lower and upper truncation values.
    
    \item The shape parameters of the kernel depend on the couple atom/task, i.e. $m\equiv m_{k,p}$ and $\sigma \equiv \sigma_{k,p}$, in order to better capture the specificity of a task and its neurological response (in time and in space).
    
    \item The truncation values $a$ and $b$ are the same for every atom $k$ and every task $p$.
    The idea behind this hypothesis is that the neurological response cannot, in any case, be shorter than $a$ (e.g., $a = \SI{30e-3}{\second}$) nor be longer than $b$ (e.g., $b = \SI{800e-3}{\second}$)
    
    \item The tasks are far enough apart from each other such that the kernel is null right before every tasks, i.e.:
    \begin{equation}\label{eq:separated_timestamps}
        t_{i+1}^{(p)} - t_i^{(p)} > b, \quad \forall p=1,\dots,P, \quad \forall i=1,\dots, (n_p-1)
    \end{equation}
    In the M/EEG context, we say that the \textit{inter stimulus interval} (ISI) is larger than $b$.
    In consequence, unlike the Hawkes processes model with exponential kernels where all past events influence the intensity, in our model only the last stimulus at time $t$ has a possible effect on the intensity at that time.
    
    Note that with this hypothesis, the intensity can be rewritten similar as a auto-regressive point process:
    \begin{equation}
        \lambda_{k,p}(t)  = \mu_k + \alpha_{k,p}\sum_{t_i^{(p)} \leq t} \kappa_{k,p}(t - t_i^{(p)})
    \end{equation}
    indeed, as $t \geq t_*^{(p)}(t) > \dots > t_2^{(p)} > t_1^{(p)}$, then $\forall i, t_i^{(p)} < t_*^{(p)}(t)$
    \begin{align*}
        t - t_i^{(p)} &= t - t_{i+1}^{(p)} + t_{i+1}^{(p)} - t_i^{(p)} \\
        &> t - t_{i+1}^{(p)} + b \\
        &\geq b, \quad \text{as } t - t_{i+1}^{(p)}\geq 0
    \end{align*}
    and as by definition, $\forall x \geq b, \kappa_{k,p}(x)=0$, thus $\kappa_{k,p}(t - t_1^{(p)}) = 0$, and finally,
    \begin{align*}
        \sum_{t_i^{(p)} \leq t} \kappa_{k,p}(t - t_i^{(p)}) &= \kappa_{k,p}(t - t_1^{(p)}) + \kappa_{k,p}(t - t_2^{(p)}) + \dots + \kappa_{k,p}(t - t_*^{(p)}(t)) \\
        &= 0 + 0 + \dots + \kappa_{k,p}(t - t_*^{(p)}(t))
    \end{align*}
    
    \item The experimentation ends after every possible neurological response driven by a task, i.e.,
    \begin{equation}\label{eq:long_duration}
        \forall p=1,\dots,P, \quad T > t_{n_p}^{(p)} + b
    \end{equation}
    $$
    \Leftrightarrow T > \maxx{p=1,\dots,P} t_{n_p}^{(p)} + b
    $$
    \item An atom is driven by only one task, i.e., we only consider the effect of one task for an atom in the model, as opposed to including several tasks.
    
    \item The baseline intensity $\mu_k$ is fixed in time.
    The baseline coefficient aims at capturing every exogenous sources of activation that are not controlled by the model, i.e., all other sources of activation apart from the considered task.
    Other sources can be other tasks, other atoms, spontaneous activations or even sources we are not aware of.
    Finally, it should be pointed out that, unlike Hawkes processes, we do not model any self-excitatory behaviour, i.e., an activation of the considered atom does not increase by itself the probability to have another activation.
\end{enumerate}

These assumptions can easily be called into question, in particular their restrictive character on the control of endogenous sources of activation, as we only consider an atom along a unique task.
That is why they we be discussed later on, and a brief overview of an extension of our model will be presented in Section~\ref{model_extension_multiple_pp}.

\subsection{Simulation}

TODO: présenter pourquoi on utilise cet algorithme, citer aussi le papier pour l'algo de simulation des processus de hawkes, et redire pourquoi ce n'est pas exactement notre cas (on est déterministe sur un des processus temporel).
Mettre l'algorithme de simulation des processus de Hawkes (thinning algo, Ogata) en annexe ?

\cite{ogata1981lewis, chen2016thinning, lewis1979simulation}

\begin{algorithm}[htpb]
\SetKw{KwDraw}{Draw}

\SetAlgoLined
\KwData{$\lambda(t)$, $T$}

initialize $n = m = 0$, $t_0 = s_0 = 0$, $\Bar{\lambda} = \max_{0 \leq t \leq T} \lambda(t)$\;
\While{$s_m \leq T$}{
   \KwDraw $u \sim \Unif{\intervalleFF{0}{1}}$\;
   $w \leftarrow -\ln u / \Bar{\lambda}$ \atcp{so that $w\sim \Exp{\Bar{\lambda}}$}
   $s_{m+1} \leftarrow s_m + w$\;
   \KwDraw $D \sim \Unif{\intervalleFF{0}{1}}$\;
   \If{$D \leq \lambda(s_{m+1}) / \Bar{\lambda}$}{
      $t_{n+1} \leftarrow s_{m+1}$\;
      $n \leftarrow n+1$\;
      }
   $m \leftarrow m+1$\;
   }
\eIf{$t_n \leq T$}{
   \Return $\braces{t_k}_{k=1,2,\dots,n}$\;
   }{
   \Return $\braces{t_k}_{k=1,2,\dots,n-1}$\;
   }
\caption{\cite{lewis1979simulation}, p.7, Algorithm 1, One-dimensional nonhomogeneous Poisson process}\label{algo:1d_inhomogenous_pp}
\end{algorithm}



\paragraph{Remark} Thanks to hypotheses \refeq{eq:m_in_between} and \refeq{eq:separated_timestamps}, it is easy to compute $\bar{\lambda}$:
\begin{equation}
    \bar{\lambda} = \max_{0\leq t \leq T} \lambda(t) = \mu + \kappa(m)
\end{equation}

TODO: calculer la complexité de cet algorithme, ou du moins ça complexité maximale, comme fait dans la thèse de Martin
\subsection{EM-based algorithm}

TODO: dire ce que l'on veut faire, Inspired by \cite{lewis2011nonparametric, xu2016learning}

\subsubsection{Computations of the coefficients update}

\paragraph{Rewriting of the kernel $\kappa$}
\begin{align*}
    \kappa(x ; m, \sigma, a, b) &= \frac{1}{\sigma} \frac{\phi\pars{\frac{x-m}{\sigma}}}{\Phi\pars{\frac{b-m}{\sigma}} - \Phi\pars{\frac{a-m}{\sigma}}} \1[a\leq x\leq b] \\
    &= \frac{1}{\sigma} \frac{\e{-\frac{1}{2} \frac{\pars{x-m}^2}{\sigma^2}}}{\integ{\frac{a-m}{\sigma}}[\frac{b-m}{\sigma}]{\e{\frac{-t^2}{2}}}{t}} \1[a\leq x\leq b] \\
    &= \frac{\e{-\frac{1}{2}\frac{\pars{x-m}^2}{\sigma^2}}}{\integ{a}[b]{\e{-\frac{1}{2}\frac{\pars{u-m}^2}{\sigma^2}}}{u}} \1[a\leq x\leq b], \quad t = \frac{u-m}{\sigma} \\
    &= \frac{\e{-\frac{1}{2}\frac{\pars{x-m}^2}{\sigma^2}}}{C\pars{m,\sigma,a,b}} \1[a\leq x\leq b]
\end{align*}
where 
\begin{align*}
    C\pars{m,\sigma,a,b} &\coloneqq \integ{a}[b]{\e{-\frac{1}{2}\frac{\pars{u-m}^2}{\sigma^2}}}{u} \\
    &= \sigma\sqrt{2\pi}\pars{\Phi\pars{\frac{b-m}{\sigma}} - \Phi\pars{\frac{a-m}{\sigma}}}
\end{align*}
and similarly, we denote by subscripts the partials derivatives:
\begin{align*}
    C_m\pars{m,\sigma,a,b} &\coloneqq \partiald{}{m} C\pars{m,\sigma,a,b} \\
    &= \integ{a}[b]{\frac{u-m}{\sigma^2}\e{-\frac{1}{2}\frac{\pars{u-m}^2}{\sigma^2}}}{u} \\
    &= \bracks{-\e{-\frac{1}{2}\frac{\pars{u-m}^2}{\sigma^2}}}_{a}^b \\
    &= \e{-\frac{1}{2}\frac{\pars{a-m}^2}{\sigma^2}} - \e{-\frac{1}{2}\frac{\pars{b-m}^2}{\sigma^2}}
\end{align*}
and 
\begin{align*}
    C_\sigma\pars{m,\sigma,a,b} &\coloneqq \partiald{}{\sigma} C\pars{m,\sigma,a,b} \\
    &= \integ{a}[b]{\frac{(u-m)^2}{\sigma^3}\e{-\frac{1}{2}\frac{\pars{u-m}^2}{\sigma^2}}}{u} \\
    &= \bracks{-\frac{u-m}{\sigma}\e{-\frac{\pars{u-m}^2}{2\sigma^2}}}_{a}^b + \frac{1}{\sigma} \integ{a}[b]{\e{-\frac{1}{2}\frac{\pars{u-m}^2}{\sigma^2}}}{u} \\
    &= \frac{a-m}{\sigma}\e{-\frac{\pars{a-m}^2}{2\sigma^2}} - \frac{b-m}{\sigma}\e{-\frac{\pars{b-m}^2}{2\sigma^2}} + \frac{1}{\sigma}C\pars{m,\sigma,a,b}
\end{align*}

\paragraph{Negative log-likelihood}
From Equation~\eqref{eq:general_likelihood}, we can define the negative log-likelihood adapted for our specific problem:
\begin{equation}
    \mathcal{L}_{k,p}\pars{\mu_k, \alpha_{k,p}, m_{k,p}, \sigma_{k,p}} \coloneqq -\log L\left(\lambda_k, \mathscr{F}_{T}^{(p)}\right) = \int_{0}^{T} \lambda_{k,p}(s) \mathrm{d} s-\sum_{t\in\mathcal{A}_k} \log \lambda_{k,p}(t)
\end{equation}

Hypothesis \eqref{eq:separated_timestamps} implies that $\forall i=1,\dots,n_p - 1$,
\begin{align*}
    \integ{t_i^{(p)}}[t_{i+1}^{(p)}]{\kappa_{k,p}(s - t_*^{(p)}(s))\1[s \geq t_1^{(p)}]}{s} &= \underbrace{\integ{t_i^{(p)}}[t_i^{(p)}+a]{\kappa_{k,p}(s - t_i^{(p)})}{s}}_{=0} + \underbrace{\integ{t_i^{(p)}+a}[t_i^{(p)}+b]{\kappa_{k,p}(s - t_i^{(p)})}{s}}_{=1} \\
    &+ \underbrace{\integ{t_i^{(p)}+b}[t_{i+1}^{(p)}]{\kappa_{k,p}(s - t_i^{(p)})}{s}}_{=0} \\
    &= 1
\end{align*}
and hypothesis \eqref{eq:long_duration} implies that
\begin{align*}
    \integ{t_{n_p}^{(p)}}[T]{\kappa_{k,p}(s - t_*^{(p)}(s))\1[s \geq t_1^{(p)}]}{s} &= \underbrace{\integ{t_{n_p}^{(p)}}[t_{n_p}^{(p)} + a]{\kappa_{k,p}(s - t_{n_p}^{(p)})}{s}}_{=0} + \underbrace{\integ{t_{n_p}^{(p)}+a}[t_{n_p}^{(p)} + b]{\kappa_{k,p}(s - t_{n_p}^{(p)})}{s}}_{=1} \\
    &+ \underbrace{\integ{t_{n_p}^{(p)}+b}[T]{\kappa_{k,p}(s - t_{n_p}^{(p)})}{s}}_{=0}\\
    &= 1
\end{align*}

Thus,
\begin{align*}
    \integ{0}[T]{\lambda_{k,p}(s)}{s} &= \integ{0}[T]{\mu_k + \alpha_{k,p}\kappa_{k,p}(s - t_*^{(p)}(s))\1[s \geq t_1^{(p)}]}{s} \\
    &= \mu_k T + \alpha_{k,p}\underbrace{\integ{0}[t_1^{(p)}]{ \kappa_{k,p}(s - t_*^{(p)}(s))\1[s \geq t_1^{(p)}]}{s}}_{=0} + \alpha_{k,p}\sum_{i=1}^{n_p -1} \integ{t_i^{(p)}}[t_{i+1}^{(p)}]{\kappa_{k,p}(s - t_i^{(p)})}{s} \\
    &+ \alpha_{k,p}\integ{t_{n_p}^{(p)}}[T]{\kappa_{k,p}(s - t_{n_p}^{(p)})}{s} \\
    &= \mu_k T + \alpha_{k,p}n_p
\end{align*}

Finally,
\begin{equation}
\begin{split}
    \mathcal{L}_{k,p}\pars{\mu_k, \alpha_{k,p}, m_{k,p}, \sigma_{k,p}} &= \mu_k T + \alpha_{k,p}n_p \\
    &- \sum_{t\in\mathcal{A}_k} \log \pars{\mu_k + \alpha_{k,p}\kappa_{k,p}(t - t_*^{(p)}(t); m_{k,p}, \sigma_{k,p})\1[t \geq t_1^{(p)}]}
\end{split}
\end{equation}

TODO : calculer la complexité du calcul de la nll, comme fait dans la thèse de Martin

Then the derivatives of the inverse log-likelihood with respect to the parameters are given by
\begin{equation}
    \partiald{}{\mu_k}\mathcal{L}_{k,p}\pars{\mu_k, \alpha_{k,p}, m_{k,p}, \sigma_{k,p}} = T - \sum_{t\in\mathcal{A}_k}\frac{1}{\lambda_{k,p}(t)}
\end{equation}

\begin{equation}
\begin{split}
    \partiald{}{\alpha_{k,p}}\mathcal{L}_{k,p}\pars{\mu_k, \alpha_{k,p}, m_{k,p}, \sigma_{k,p}} &= n_p - \sum_{t\in\mathcal{A}_k}\frac{\kappa_{k,p}(t - t_*^{(p)}(t))}{\lambda_{k,p}(t)}\1[t \geq t_1^{(p)}] \\
    &= n_p - \sum_{t\in\mathcal{A}_k, t \geq t_1^{(p)}}\frac{\kappa_{k,p}(t - t_*^{(p)}(t))}{\lambda_{k,p}(t)}
\end{split}
\end{equation}

\begin{equation}
\begin{split}
    \partiald{}{m_{k,p}}\mathcal{L}_{k,p}\pars{\mu_k, \alpha_{k,p}, m_{k,p}, \sigma_{k,p}} &= -\sum_{t\in\mathcal{A}_k}\frac{\alpha_{k,p}}{\lambda_{k,p}(t)}\partiald{}{m_{k,p}}\kappa_{k,p}(t - t_*^{(p)}(t); m_{k,p}, \sigma_{k,p}, a, b)\1[t \geq t_1^{(p)}] \\
    &= -\sum_{t\in\mathcal{A}_k, t \geq t_1^{(p)}}\frac{\alpha_{k,p}}{\lambda_{k,p}(t)}\partiald{}{m_{k,p}}\kappa_{k,p}(t - t_*^{(p)}(t); m_{k,p}, \sigma_{k,p}, a, b)
\end{split}
\end{equation}
where
\begin{equation}
    \partiald{}{m}\kappa(x ; m, \sigma, a, b) = \pars{\frac{x-m}{\sigma^2} - \frac{C_m\pars{m, \sigma,a,b}}{C\pars{m, \sigma,a,b}}}\kappa(x ; m, \sigma, a, b)
\end{equation}

\begin{equation}
\begin{split}
    \partiald{}{\sigma_{k,p}}\mathcal{L}_{k,p}\pars{\mu_k, \alpha_{k,p}, m_{k,p}, \sigma_{k,p}} &= -\sum_{t\in\mathcal{A}_k}\frac{\alpha_{k,p}}{\lambda_{k,p}(t)}\partiald{}{\sigma_{k,p}}\kappa_{k,p}(t - t_*^{(p)}(t); m_{k,p}, \sigma_{k,p}, a, b)\1[t \geq t_1^{(p)}] \\
    &= -\sum_{t\in\mathcal{A}_k, t \geq t_1^{(p)}}\frac{\alpha_{k,p}}{\lambda_{k,p}(t)}\partiald{}{\sigma_{k,p}}\kappa_{k,p}(t - t_*^{(p)}(t); m_{k,p}, \sigma_{k,p}, a, b)
\end{split}
\end{equation}
where
\begin{equation}
    \partiald{}{\sigma}\kappa(x ; m, \sigma, a, b) = \pars{\frac{(x-m)^2}{\sigma^3} - \frac{C_\sigma\pars{m, \sigma,a,b}}{C\pars{m, \sigma,a,b}}}\kappa(x ; m, \sigma, a, b)
\end{equation}

To lighten notations, we define $\mathcal{A}_{k,p} = \enstq{t}{t\in\mathcal{A}_k, t \geq t_1^{(p)}}$ as the set of atom $k$'s activation timestamps that occur after the first timestamps of task $p$.
In other words, $\mathcal{A}_{k,p}$ is the set of all atom $k$'s activations that might have been driven by a task of type $p$.

\paragraph{Expectation step}
Let $P_{t,k}$ be the probability that the activation at time $t$ has been triggered by the baseline intensity of atom $k$, and $P_{t,p}$ be that probability that the activation at time $t$ has been triggered by the driver $p$ (the task $p$ in our case).
\begin{equation}
    P_{t,k} = \frac{\mu_k}{\lambda_{k,p}(t)}
\end{equation}
\begin{equation}
    P_{t,p} = \frac{\alpha_{k,p}\kappa_{k,p}\pars{t - t_*^{(p)}(t)}}{\lambda_{k,p}(t)}\1[t \geq t_1^{(p)}]
\end{equation}

\paragraph{Maximisation step} Details of the computation are present in Appendix~\ref{annexe:details_em}.

\begin{equation}
    \mu_k^{(n+1)} = \frac{1}{T} \sum_{t\in\mathcal{A}_k} P_{t,k}^{(n)}
\end{equation}

\begin{equation}
    \alpha_{k,p}^{(n+1)} = \frac{1}{n_p} \sum_{t\in\mathcal{A}_{k,p}} P_{t,p}^{(n)}
\end{equation}

\begin{equation}
    m_{k,p}^{(n+1)} = \frac{\sum_{t\in\mathcal{A}_{k,p}} \pars{t - t_*^{(p)}(t)} P_{t,p}^{(n)}}{\sum_{t\in\mathcal{A}_{k,p}} P_{t,p}^{(n)}} - {\sigma_{k,p}^{(n)}}^2 \frac{C_m\pars{m_{k,p}^{(n)}, \sigma_{k,p}^{(n)},a,b}}{C\pars{m_{k,p}^{(n)}, \sigma_{k,p}^{(n)},a,b}}
\end{equation}

\begin{equation}
    \sigma_{k,p}^{(n+1)} = \pars{\frac{C\pars{m_{k,p}^{(n)}, \sigma_{k,p}^{(n)},a,b}}{C_\sigma\pars{m_{k,p}^{(n)}, \sigma_{k,p}^{(n)},a,b}} \frac{\sum_{t\in\mathcal{A}_{k,p}} \pars{t - t_*^{(p)}(t) - m_{k,p}}^2 P_{t,p}^{(n)}}{\sum_{t\in\mathcal{A}_{k,p}}  P_{t,p}^{(n)}}}^{1/3}
\end{equation}

\paragraph{Remark} If $\alpha_{k,p} = 0$, then the intensity is reduced to its baseline, $\lambda_{k,p}(t) = \mu_k$, and then the negative log-likelihood is:
\begin{equation}
    \mathcal{L}_{k,p}\pars{\mu_k, \alpha_{k,p}, m_{k,p}, \sigma_{k,p}} = \mu_k T -\sum_{t\in\mathcal{A}_k} \log \mu_k = \mu_k T - \#\mathcal{A}_k \log \mu_k
\end{equation}
Thus, we can directly compute the maximum likelihood estimator (MLE) for $\mu_k$, as follow:
\begin{equation}
    \mu_k^{(MLE)} = \frac{\#\mathcal{A}_k}{T}
\end{equation}

\subsubsection{Parameters initialisation}

\paragraph{Random initialisation}

TODO

\paragraph{Smart initialisation}

TODO

More details on the role of the $\alpha$ coefficient and how its initialisation function was determined are presented in Appendix~\ref{app:role_alpha_coef}



\subsection{Model extension to multiple point processes drivers}\label{model_extension_multiple_pp}

In the above, the model presented considered only one potential \say{driver}, as for every atom, the intensity $\lambda_{k,p}$ was only influenced by the stimulus $p$.
However, this assumption may be somewhat too restrictive.
Indeed, it is quite possible that an atom $k$ to be stimulated by both stimuli $p$ and $p'$ at the same time.
A first solution simply consists in combining both $p$ and $p'$ timestamps by defining $t^{(p, p')} \coloneqq t^{(p)} \cup t^{(p')}$, and thus,
\begin{equation}\label{eq:intensity_combined_stimuli}
    \lambda_{k,(p, p')}(t)  = \mu_k + \alpha_{k,(p, p')}\kappa_{k,(p, p')}(t - t_*^{(p, p')}(t))\1[t \geq t_1^{(p, p')}]
\end{equation}
This is this straightforward solution that is used in this work, and can easily be generalised to 3 or more \say{drivers}.

However, the main flaw of this solution is that it blurs the individual effect of each stimulus on the considered atom.
That is why when this solution is used, it is with done with stimuli that lead to similar reactions in the brain, such as two auditory stimuli, one on the left ear and the other on the right ear.
A proposed solution, which it will be interesting to deal with more in depth in future work, is to take into the intensity function multiple \say{drivers} while keeping them separated in order to be able to distinguish their respective effect on the intensity, by no longer having a single coefficient $\alpha$ for this purpose.
Let $\mathcal{P}$ be a (non-empty) set a drivers, e.g. $\mathcal{P} = \braces{p, p'}$, thus, the new intensity function can be written as follow:
\begin{equation}
    \lambda_{k,\mathcal{P}}(t)  = \mu_k + \sum_{p\in\mathcal{P}} \alpha_{k,p}\kappa_{k,p}(t - t_*^{(p)}(t))\1[t \geq t_1^{(p)}]
\end{equation}

It is possible to go even further by considering that the considered atom might eventually be influenced by the activation of other atoms, while keeping the assumption that it does not have a self-excitatory behaviour.
Hence, the intensity function must include terms for such an influence.
Let $\mathcal{K}$ be a (non-empty) set of atoms, thus, the new intensity function can be written as follow:
\begin{equation}
    \lambda_{k,\mathcal{K},\mathcal{P}}(t)  = \mu_k + \sum_{p\in\mathcal{P}} \alpha_{k,p}\kappa_{k,p}(t - t_*^{(p)}(t))\1[t \geq t_1^{(p)}] + \sum_{k'\in\mathcal{K}, k' \ne k} \beta_{k,k'} \varphi_{k,k'}\pars{t - \mathcal{A}_{k'}^*(t)}
\end{equation}
where $\mathcal{A}_{k'}^*(t) \coloneqq \max\enstq{t'}{t' \in \mathcal{A}_{k'}, t' \leq t}$ denotes the timestamp of the last activation on atom $k'$ at time $t$, $\beta_{k,k'}\in\R$ a coefficient taht allow to quantify the influence of atom $k'$ on atom $k$, and where $\varphi_{k,k'}$ is a kernel function to be specified.
Note that the final idea is that $\mathcal{P}$ and $\mathcal{K}$ are the entire set of stimuli and atoms, respectively.

The assumption of no self-excitatory behaviour can also be easily waived, thus the intensity function is only slightly modified, as follow:
\begin{equation}
    \lambda_{k,\mathcal{K},\mathcal{P}}(t)  = \mu_k + \sum_{p\in\mathcal{P}} \alpha_{k,p}\kappa_{k,p}(t - t_*^{(p)}(t))\1[t \geq t_1^{(p)}] + \sum_{k'\in\mathcal{K}} \beta_{k,k'} \varphi_{k,k'}\pars{t - \mathcal{A}_{k'}^*(t)}
\end{equation}
Note that in that case, if $k\in\mathcal{K}$, then the kernel $\varphi_{k,k}$ characterises this newly self-excitatory behaviour.

As mentioned, those new approaches are not developed nor used later on in this report and will be the subject of future work.
We only introduced them for the sake of completeness.
