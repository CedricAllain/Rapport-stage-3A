\subsection{Simulation}

As mentioned before, the intensity for an atom $k$ modelled with a specific stimulus $p$ only depends on a baseline intensity that is fixed in time and on a term independent from the activations of the atom.
Thus, as $e_p$ is deterministic, at every time $t \in \intervalleFF{0}{T}$, $\lambda_{k,p}(t)$ takes a deterministic value.
As a result, it is possible to consider the temporal point process $z_k$ as an inhomogenous Poisson process, as its intensity function is deterministic and non-constant in time.
For those reasons, in order to simulate $z_k$ data, we simply use the one-dimensional nonhomogeneous Poisson process algorithm described in \citep{lewis1979simulation, chen2016thinning} and detailed in Algorithm~\ref{algo:1d_inhomogenous_pp}.

Starting from $s_0=0$, at each iteration the algorithm generates a new event candidate $s_{m+1}$ sampled from a homogeneous Poisson process of constant intensity $\bar{\lambda} \coloneqq \max_{0\leq t \leq T} \lambda(t)$.
This candidate is either accepted with probability $\lambda(s_{m+1})/\Bar{\lambda}$ or discarded.

Note that the self-excitatory behaviour and the multi-directional influence present in Hawkes processes model, but assumed to be non-existent in ours, make the simulation algorithm more complex, as mentioned in \citep{ogata1981lewis, chen2016thinning, bompaire2019machine}.
Thus, if the assumptions concerning these two behavioural characteristics are amended, the simulation algorithm must be changed accordingly.

\begin{algorithm}[htpb]
\SetKw{KwDraw}{Draw}

\SetAlgoLined
\KwData{$\lambda(t)$, $T$}

initialize $n = m = 0$, $t_0 = s_0 = 0$, $\Bar{\lambda} = \max_{0 \leq t \leq T} \lambda(t)$\;
\While{$s_m \leq T$}{
   \KwDraw $u \sim \Unif{\intervalleFF{0}{1}}$\;
   $w \leftarrow -\ln u / \Bar{\lambda}$ \atcp{so that $w\sim \Exp{\Bar{\lambda}}$}
   $s_{m+1} \leftarrow s_m + w$\;
   \KwDraw $D \sim \Unif{\intervalleFF{0}{1}}$\;
   \If{$D \leq \lambda(s_{m+1}) / \Bar{\lambda}$}{
      $t_{n+1} \leftarrow s_{m+1}$\;
      $n \leftarrow n+1$\;
      }
   $m \leftarrow m+1$\;
   }
\eIf{$t_n \leq T$}{
   \Return $\braces{t_k}_{k=1,2,\dots,n}$\;
   }{
   \Return $\braces{t_k}_{k=1,2,\dots,n-1}$\;
   }
\caption{\cite{lewis1979simulation}, p.7, Algorithm 1, One-dimensional nonhomogeneous Poisson process}\label{algo:1d_inhomogenous_pp}
\end{algorithm}


\paragraph{Remark} Thanks to hypotheses \refeq{eq:m_in_between} and \refeq{eq:separated_timestamps}, it is easy to compute $\bar{\lambda}$:
\begin{equation}\label{eq:max_lambda}
    \bar{\lambda} = \max_{0\leq t \leq T} \lambda(t) = \mu + \kappa(m)
\end{equation}
Thus, it is easy to see that more events, atom's activation in our case, will be generated around times $t = t_i^{(p)} + m, i=1,\dots,n_p$ as the intensity is maximal at these points.
