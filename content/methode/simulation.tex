\subsection{Simulation}

TODO: présenter pourquoi on utilise cet algorithme, citer aussi le papier pour l'algo de simulation des processus de hawkes, et redire pourquoi ce n'est pas exactement notre cas (on est déterministe sur un des processus temporel).
Mettre l'algorithme de simulation des processus de Hawkes (thinning algo, Ogata) en annexe ?

\cite{ogata1981lewis, chen2016thinning, lewis1979simulation}

\begin{algorithm}[htpb]
\SetKw{KwDraw}{Draw}

\SetAlgoLined
\KwData{$\lambda(t)$, $T$}

initialize $n = m = 0$, $t_0 = s_0 = 0$, $\Bar{\lambda} = \max_{0 \leq t \leq T} \lambda(t)$\;
\While{$s_m \leq T$}{
   \KwDraw $u \sim \Unif{\intervalleFF{0}{1}}$\;
   $w \leftarrow -\ln u / \Bar{\lambda}$ \atcp{so that $w\sim \Exp{\Bar{\lambda}}$}
   $s_{m+1} \leftarrow s_m + w$\;
   \KwDraw $D \sim \Unif{\intervalleFF{0}{1}}$\;
   \If{$D \leq \lambda(s_{m+1}) / \Bar{\lambda}$}{
      $t_{n+1} \leftarrow s_{m+1}$\;
      $n \leftarrow n+1$\;
      }
   $m \leftarrow m+1$\;
   }
\eIf{$t_n \leq T$}{
   \Return $\braces{t_k}_{k=1,2,\dots,n}$\;
   }{
   \Return $\braces{t_k}_{k=1,2,\dots,n-1}$\;
   }
\caption{\cite{lewis1979simulation}, p.7, Algorithm 1, One-dimensional nonhomogeneous Poisson process}\label{algo:1d_inhomogenous_pp}
\end{algorithm}



\paragraph{Remark} Thanks to hypotheses \refeq{eq:m_in_between} and \refeq{eq:separated_timestamps}, it is easy to compute $\bar{\lambda}$:
\begin{equation}
    \bar{\lambda} = \max_{0\leq t \leq T} \lambda(t) = \mu + \kappa(m)
\end{equation}

TODO: calculer la complexité de cet algorithme, ou du moins ça complexité maximale, comme fait dans la thèse de Martin