\usepackage{amsthm}
%\theoremstyle{theorem}
%\newtheorem{theorem}{Théorème}[chapter]

%\theoremstyle{theorem}
%\newtheorem{prop}{Propriété}[chapter]

%\theoremstyle{theorem}
%\newtheorem{proposition}{Proposition}[chapter]

%\theoremstyle{theorem}
%\newtheorem{corollaire}{Corollaire}[chapter]

%\theoremstyle{lemma}
%\newtheorem{lemma}{Lemme}[chapter]

%\theoremstyle{definition}
%\newtheorem{definition}{Définition}[chapter]

%\theoremstyle{exemple}
%\newtheorem{exemple}{Exemple}[chapter]

%\theoremstyle{remarque}
%\newtheorem{remarque}{Remarque}[chapter]

\usepackage{amsmath}        % La base pour les maths
\usepackage{mathrsfs}       % Quelques symboles supplémentaires
\usepackage{amssymb}        % encore des symboles.
\usepackage{amsfonts}       % Des fontes, eg pour \mathbb.
\usepackage{mathtools}      % Permet d'utiliser des symboles d'égalités spéciaux
\usepackage{centernot}
\usepackage{dsfont}         % Pour avoir les notations nécessaires aux ensembles (nécessite mathbb)
\usepackage{stmaryrd}       % Permet de faire des intervalles avec des doubles barres (intervalles d'entiers)
\usepackage{esvect}         % Donne plein de possibilité pour les flèches des vecteurs
\usepackage{systeme}        % Permet de créer facilement des systèmes (en mode mathématique ou en texte)