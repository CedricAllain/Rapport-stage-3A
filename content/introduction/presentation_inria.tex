\subsection{Inria and Parietal team}

I did my internship at Inria Saclay, within the Parietal team, under the supervision of \href{http://alexandre.gramfort.net}{Alexandre Gramfort}, senior research scientist, and \href{https://tommoral.github.io/about.html}{Thomas Moreau}, research scientist, both permanent members of the Parietal team.
This section aims to briefly present the research institute of Inria, with a focus on the Parietal team.

\paragraph{Inria} The \href{https://www.inria.fr/fr}{Institute for Research in Computer Science and Automatic} (Inria\footnote{Institut de recherche en informatique et en automatique}) is a national institute for research in digital science and technology.
It has been funded in 1967, within the framework of the ``\textit{Plan Calcul}''\footnote{The Plan Calcul was a French governmental program launched in 1966 by President Charles de Gaulle designed to ensure the country's autonomy in information technology, and to develop European IT.}, and employs \num{3500} researchers and engineers often working in an interdisciplinary manner and in collaboration with industrial partners.
Inria Saclay is one of the 9 research centres spread over the territory that compose Inria.
 Also, thanks to \href{https://learninglab.inria.fr}{Inria Learning Lab}, Inria designs MOOCs to disseminate knowledge in digital sciences and strengthen the dialogue between science and society.


\paragraph{Parietal team} Within Inria, reasearchers are divided among \num{200} project-teams, the \href{https://team.inria.fr/parietal/}{Parietal team} being one of them.
It is an Inria-CEA joint team part of the Neurospin research centre, headed by \href{https://pages.saclay.inria.fr/bertrand.thirion/}{Bertrand Thirion}, that focuses on mathematical methods for statistical modelling of brain function using neuroimaging data (fMRI, MEG, EEG), with a particular interest in machine learning techniques, applications to human cognitive neuroscience, and scientific software development\footnote{Source: presentation of Parietal, \url{https://team.inria.fr/parietal/}}.
The different members of Parietal are committers to a variety of open-source projects such as Scikit-Learn, 
NiLearn and MNE-Python, among others.
