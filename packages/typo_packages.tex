%\usepackage{numprint}  % Histoire que les chiffres soient bien
\usepackage{eurosym}  % Permet l'utilisation du signe €
% Exemple : \EUR{\num{299792458.38}}

\usepackage[maxlevel=3]{csquotes}  % Propose des commandes pour les citations. L'option maxlevel permet de déterminer le nombre de niveaux d'imbrication maximal.
% Exemple :  \enquote{Lorsque yyy déclare \enquote{zzz} il ne déclare rien du tout}. Les citations en bloc sont également possible : \begin{quote} citation \end{quote}. Pour les citations plus longues, on préfèrera \begin{quotation} citation longue \end{quotation}. 

\usepackage{setspace}  % Permet de modifier l'espacement entre les lignes.
%\singlespacing  % interligne simple
%\onehalfspacing  % interligne 1.5
\doublespacing  % double interligne 

\usepackage{siunitx}  % Pareil que numprint, mais pour un cadre mathématiques, donc permet de faire plus de chose : http://ctan.mines-albi.fr/macros/latex/contrib/siunitx/siunitx.pdf

\sisetup{locale = UK} % Permet de selectionner le type de convention (UK, US, FR)
% Exemples :
%\num{299792458} :  Il y a des espaces.
%\num{299792,458} : On obtient une virgule comme séparateur décimal.
%\num{.34} : Même en utilisant le point, on obtient une virgule et un zéro est placé.
%\num{299792,458 +- 0.09} : on peut obtenir le signe « plus ou moins » avec +-.
%\num{3e8} : « e » permet d’obtenir la puissance de 10, avec le signe x (3 x 10^8)

%\usepackage[
%    left = \flqq{},% 
%    right = \frqq{},% 
%    leftsub = \flq{},% 
%    rightsub = \frq{} %
%]{dirtytalk}
\usepackage{dirtytalk}