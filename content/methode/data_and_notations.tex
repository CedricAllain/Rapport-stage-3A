\subsection{Data and notations}

TODO : Faire la distinction entre le modèle théorique (processus guidé et processus "guideur"), et l'application : les z\_k sont les processus guidés, et dans l'application on les obtient de telle manière [...] 

As explained before, the \textit{driven temporal point processes} approach consists in two sets of timestamps, one for 

TODO: dans l'appllication M/EEG, dire que les multiples signaux reçus par les capteurs sont décomposés en K recurring patterns, où K est un hyperparamètre (TODO: dans les fais, comment est déterminé K, est-ce qu'on donne un nombre assez élevé afin d'être sûr de capter tous les atomes intéressants ?) et que l'on fait plusieurs type de stimulis, avec exemples : 

\begin{itemize}
    \item $\braces{z_k}_{k=1,\dots,K}$: the family of the $K$ atoms' activation vector (an example of an activation vector is shown in Figure~\ref{fig:signal_decomposition}).
    ${\forall k=1,\dots,K, z_k\in\R_+^\mathbf{T}}$, where $\mathbf{T}$ is the total number of possible timestamps.
    $\mathbf{T} \coloneqq \ent{T \times s_{freq}}$, where $T$ is the total duration of the experimentation (e.g., $T = \SI{3}{\minute}$), and $s_{freq}$ is the sampling frequency for the collect of MEG data (e.g., $s_{freq} = \SI{150}{\per\second}$), and where $\ent{x}$ denotes the bigger integer smaller or equal to $x$.
    We define $\mathcal{A}_k \coloneqq \enstq{t\times s_{freq}^{-1}}{z_k(t)\geq\tau, t=1,\dots, \mathbf{T}}$ the set of all the timestamps of atom $k$'s activations, where $\tau$ is a pre-determined threshold used to filter out insignificant activations (e.g., $\tau = \num{0.7e-10}$).
    
    \item $\braces{e_p}_{p=1,\dots,P}$: the family of the $P$ types of tasks performed during the experimentation
    \begin{equation*}
        \forall t=1,\dots,\mathbf{T}, \quad e_p(t)=
        \left\{
		\begin{array}{ll}
			1 & \mbox{if a task of type $p$ has occurred at the $t$-th timestamp}\\
			0 & \mbox{otherwise}
		\end{array}
	\right.
    \end{equation*}
    
    \item $t^{(p)} \coloneqq \enstq{t\times s_{freq}^{-1}}{e_p(t)=1, t=1,\dots, \mathbf{T}}$ denotes all the timestamps when a task of type $p$ occurred.
    We thus have $t^{(p)} = \braces{t^{(p)}_1, \dots, t^{(p)}_{n_p}}$ sorted, i.e., $t^{(p)}_1 < t^{(p)}_2 < \dots < t^{(p)}_{n_p}$, where $n_p=\# t^{(p)}$ is the total number of task $p$ that has occurred during the experimentation, as $\# A$ denotes the cardinality of the set $A$.
\end{itemize} 

TODO: afin de préciser une dernière fois, les $e_p$ sont les determinist point process, qui drivent les $z_k$.
Dans les faits, à la fin de l'expérience, on a toutes les données, les vecteurs $z_k$ sont tous connus.
Notre but est alors de déterminer si, pour un coupe $(e_p, z_k)$ donné, si le vecteur d'activation a été influencé au stimulus, et donc si l'atome k est "lié" au stimulus p.
A noter que l'on ne peux exhiber qu'une relation de corrélation et non pas de causalité, notamment dû au fait que l'on ne contrôle pas tout l'environnement, il peut donc exister des facteurs exhogènes non contrôlés par le modèle qui peuvent être à l'origine des activations sur $z_k$, et non pas $e_p$.
Ainsi, dans un premmier temps, on ne considère un atome qu'à la lumière d'un seul stimulus à la fois.